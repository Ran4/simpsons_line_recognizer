
\documentclass{beamer}
\usepackage{mathptmx}
\usepackage{wasysym}
\usepackage[11pt]{moresize}
\usepackage[swedish]{babel}
\usepackage{synttree}
\usepackage{listings}
\usepackage[utf8]{inputenc}
\usepackage[T1]{fontenc}
\usepackage{graphicx}
\date{\today}
\lstdefinestyle{basic}{
    captionpos=t,%
    basicstyle=\footnotesize\ttfamily,%
    numberstyle=\tiny,%
    numbers=left,%
    stepnumber=1,%
    frame=single,%
    showspaces=false,%
    showstringspaces=false,%
    showtabs=false,%
    %
    keywordstyle=\color{blue},%
    identifierstyle=,%
    commentstyle=\color{gray},%
    stringstyle=\color{magenta}%
}
\title{Replikidentifiering av karaktärer från tv-serien The Simpsons med N-Grams}
\author{Anton Kindestam antonki@kth.se\\Rasmus Ansin ransin@kth.se}
\lstset{basicstyle=\small,breaklines=true, breakatwhitespace=false, showspaces=false, showstringspaces=false}

\begin{document}


\frame{\titlepage}






\renewcommand{\thefootnote}{\fnsymbol{footnote}} % A sequence of nine symbols (try it and see!)


%==Replikidentifiering av karaktärer från tv-serien The Simpsons med N-Grams==

\section{Dyker den här texten ens upp?}



\begin{frame}
 \frametitle{Bakgrund}
  


Vi har valt att analysera repliker av The Simpsons-avsnitt för att
skapa en klassificerare som kan klassificera vilken yttring som
tillhör vilken Simpsons-rollfigur. Vi gör detta genom att skapa en
n-gram-databas över yttringar kopplade till rollfigurer. För en given
mening generar vi dess n-gram, och sen väljer vi den rollfigur för
vilken denna n-gram-sekvens är sannolikast.

  
\end{frame}
\begin{frame}
 \frametitle{Parsning av Data}
  


\begin{itemize}
  \item Blah1
\begin{itemize}
  \item Blah1.1
\end{itemize}
  \item Blah2
\end{itemize}


  
\end{frame}
\begin{frame}
 \frametitle{Vår Modell}
  





  
\end{frame}


\end{document}

